\documentclass[12pt,a4paper]{article}
\usepackage{amsmath,amssymb,bm,geometry}
\geometry{margin=2.5cm}
\begin{document}

\begin{center}
    {\Large\bf Words as Sources: Entropy and Redundancy in the Embedding Space}\\[0.5em]
\end{center}

Language can be seen as a great communication network whose nodes are words, each acting as a {\it source of information}.  
They can also be seen as outputs of world-word mappings, where each word encodes a concept or object in the external world.
In this case we may see their semantic contentn as a signal transmitted through a noisy channel.

In NLP representation learning, the meaning is formalized as a {\it word embedding}---a point 
$\bm{w}_i \in \mathbb{R}^d$---a small coordinate in a high-dimensional semantic manifold.  
The embedding compresses the uncertainty of linguistic form into the geometry of meaning.  
If two words, such as \emph{big} and \emph{large}, fall into the same region of this space, 
they share an information channel; their redundancy mirrors synonymy.

\vspace{1em}

Imagine binning the embedding space into cells $\{ \mathcal{E}_1, \ldots, \mathcal{E}_K \}$, 
as if discretizing a continuous signal.  
Each cell collects all words whose vectors fall within it, and therefore represents a 
coarse semantic class.  
If we define
\begin{equation}
p(B=k) = \frac{1}{|V|}\sum_{i=1}^{|V|} \mathbf{1}_{\bm{w}_i \in \mathcal{E}_k},
\end{equation}
then the entropy of the binned representation,
\begin{equation}
H(B) = - \sum_{k=1}^{K} p(B=k)\,\log p(B=k),
\end{equation}
measures how widely dispersed meanings are across the semantic manifold.  
Low entropy indicates tight clustering---a language rich in redundancy;  
high entropy signals a lexicon stretched across conceptual space.

\vspace{1em}

From an information-theoretic perspective, we can regard the embedding function
\begin{equation}
f : V \rightarrow \mathbb{R}^d, \qquad f(w_i)=\bm{w}_i,
\end{equation}
as an {\it encoder} in a semantic channel:
\[
w_i \longrightarrow \bm{w}_i \longrightarrow \hat{w}_i .
\]
Here the discrete lexical source $W$ emits symbols $w_i$ with probability $p(w_i)$, 
the encoder maps them to continuous signals $\bm{w}_i$, 
and a decoder attempts to reconstruct the intended concept $\hat{w}_i$.  
The process carries mutual information
\begin{equation}
I(W;\bm{W}) = H(W) - H(W\mid \bm{W}),
\end{equation}
quantifying how much of the lexical uncertainty is preserved in the embedding geometry.  
When two words have nearly identical embeddings, 
the channel becomes degenerate: the same semantic code corresponds to multiple lexical sources, 
and synonymy appears as a natural form of compression.

\vspace{1em}

In this view, the embedding space is not merely a storage of vectors but a 
semantic communication medium.  
Each word is a transmitter, each embedding a signal, and each synonym a sign of the system's efficiency---  
a reminder that language, like any channel, trades entropy for meaning.

TODO:
\paragraph*{Relation to Carnap–Bar-Hillel}
\begin{itemize}
\item Unit: continuous word embeddings (vectors) vs logical sentences/propositions over state-descriptions.
\item Measure: Shannon entropy/mutual information in a communication channel vs logical content \(I(\varphi) = -\log P(\varphi)\).
\item Probabilities: corpus- and task-induced empirical distributions vs a priori logical probability.
\item Structure: geometric similarity and redundancy in vector space vs set-theoretic entailment and inclusion.
\item Outcomes: channel efficiency and compression (synonymy as redundancy) vs semantic content and confirmation; no analogue of the Bar-Hillel–Carnap paradox (tautology=0, contradiction=max).
\end{itemize}


N-grams 
if ... then .... phrases (as vectors of words  with hard scaffolding) 
how to encode them hard 
negation as additional token for all words (as one position of words negated, given negation of sentence negates ONLYone word )
Describe shortly 




\end{document}
